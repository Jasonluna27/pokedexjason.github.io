%Preambulo
\documentclass[12pt]{article}
\usepackage[utf8]{inputenc}
\usepackage[spanish]{babel}
\usepackage{geometry,graphicx}

%Tamaño de Hoja y Márgenes del Documento
\geometry{letterpaper, left=2cm,right=2cm,top=2cm,bottom=4cm}


%Inicio del documento
\begin{document}
	%Portada
	\begin{titlepage}
		\centering
		\textbf{\LARGE Universidad de Costa Rica}\\
		\vspace{3cm}
		\textbf{\Large Ingeniería Eléctrica}\\
		\vspace{1.5cm}
		\textsc{ \Huge Proyecto:}\\
		\vspace{3cm}
		\textit{\Large Pokédex en Javascript }\\
		\vfill       
		\Large Autor: \\
		\Large Jason Antonio Luna Vega\\
		\vfill
		\Large Octubre 2023\\
	\end{titlepage}
	
	\tableofcontents\vspace{18cm}

    \section{Introducción: }
    En este documento se detallará el proyecto de programación que se desarrollará para este curso, el cual consiste en crear una aplicación web basada en una pokédex la cual es un sistema creado en el mundo de pokémon para poder llevar un inventario de la cantidad de pokémones capturados a lo largo del juego.\vspace{0.5cm}
    \\
    Para este proyecto se hará uso de tecnologías como HTML,CSS,JavaScript,ReactJs,Github,GithubPages y la API de PokéApi de cual se extraerá la información de cada uno de los pokémones que se podrán consultar en la aplicación.\vspace{0.5cm}
    \\

    \section{Objetivos: }
        \begin{itemize}
            \item Objetivo General: Desarrollar una aplicación web en JavaScript basada en el sistema de Pokédex donde los usuarios puedan consultar la información de los pokémons por nombre, tipo, región y generación.
            \item Objetivos Especificos:
                \subitem Implementar un API en la aplicación que facilite la obtención de datos.
                \subitem Desplegar la aplicación para el uso del publico.
                \subitem Implementar un sistema de control de versiones para el desarrollo de la aplicación.
                \subitem Hacer uso de frameworks los cuales faciliten la creación de aplicaciones web.
        \end{itemize}

    \section{Alcances:}
        \begin{itemize}
            \item Aplicación funcional.
            \item Amigable con el usuario.
            \item Interfaz facil de usar.
            \item Acceso al público.
            \item Despliegue vía web.
            \item Manejo correcto de la información.
            \item Diseño responsivo.
        \end{itemize}

    \section{Justificación: }
    El motivo del desarrollo de este proyecto es debido a que siempre me ha gustado este tipo de juegos pero es un poco molesto tener que capturar los pokémons para visualizar su información o saber a que posición de la pokédex pertenece un pokémon por lo que lo mas convencional es tener un sistema donde se pueda consultar los datos de cada uno sin necesidad de tener que capturarlos dentro del juego.\vspace{0.5cm}
    \\
    De esta forma los usuarios pueden tomar mejores decisiones a la hora de formar su equipo dentro del juego cuales son los pokémons que mas les convienen o que mas les gusten. También saber donde se encuentra cada uno de estos dentro del juego, saber su tipo y demás detalles que son importantes a la hora de avanzar en la historia del juego.\vspace{0.5cm}
    \\

    \section{Marco Teórico: }
        \subsection{Antecedentes:}
            Este es un proyecto desarrollada anteriormente por multiples personas por lo cual se cuenta con una amplia variedad de ejemplos de los cuales se tomarán como referencia para la realización de este sistema esto no quiere decir que se vaya a copiar de un ejemplo especifico si no usar todas estas referencias para desarrollar una aplicación que cumpla con los objetivos ya descritos anteriormente.
        \subsection{Problema:}
            El problema que se quiere desarrollar y resolver es el proceso tardado de cada usuario al estar jugando un juego de pokémon y tener que capturar cada uno de los pokémons para obtener la información de cada uno de ellos lo cual es un proceso muy tardado, con el sistema de pokédex se quiere facilitar esta labor a los entrenadores y de esta forma ellos puedan encontrar los pokémons que mas les convenga para su aventura.
        \subsection{Requisitos del proyecto: }
            \begin{itemize}
                \item Debe permitir a los usuarios consultar los pokémons por su nombre, tipo, región y generación.
                \item Interfaz amigable para el usuario y que pueda ser utilizada en cualquier dispositivo.
                \item Acceso al público en general para el uso del sistema.
            \end{itemize}
            \vspace{3cm}
        \subsection{Estructura del proyecto: }
            \begin{itemize}
                \item Capa de presentación: Esta capa se encargará de mostrar la interfaz de usuario del sistema. Estará construida con HTML, CSS y JavaScript.
                \item Capa de procesamiento de datos: Esta capa se encargará de procesar las peticiones de los usuarios y devolver la información correspondiente. Estará construida con JavaScript y ReactJS.
            \end{itemize}
        \subsection{Metodología de desarrollo: }
        El proyecto se desarrollará utilizando la metodología de desarrollo ágil. Esta metodología se basa en el desarrollo iterativo e incremental, lo que significa que el sistema se irá construyendo por partes, de forma que se pueda ir evaluando su progreso a medida que se desarrolla.
        \subsection{Tecnologías: }
        El proyecto se desarrollará utilizando las siguientes herramientas y tecnologías:
        \begin{itemize}
            \item HTML: El lenguaje de marcado para la estructura de la página web.
            \item CSS: El lenguaje de hojas de estilo para el diseño de la página web.
            \item JavaScript: El lenguaje de programación para la lógica del sistema.
            \item ReactJS: Una biblioteca de JavaScript para la creación de interfaces de usuario.
            \item GitHub: Un sistema de control de versiones para el seguimiento del código.
            \item GitHub Pages: Un servicio de alojamiento web para la publicación de proyectos de GitHub.
        \end{itemize}
        \subsection{title}
\end{document}